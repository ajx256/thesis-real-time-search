%%%%%%%%%%%%%%%%%%%%%%%%%%%%%%%%%%%%%%%%%%%%%%%%%%%%%%%%%%%%%%%%%%%%%%%%%%%%%%%%
%2345678901234567890123456789012345678901234567890123456789012345678901234567890
%        1         2         3         4         5         6         7         8

\documentclass[letterpaper, 11 pt, conference]{ieeeconf}  % Comment this line out
% if you need a4paper
%\documentclass[a4paper, 10pt, conference]{ieeeconf}      % Use this line for a4
% paper
% See the \addtolength command later in the file to balance the column lengths
% on the last page of the document



% The following packages can be found on http:\\www.ctan.org
%\usepackage{graphics} % for pdf, bitmapped graphics files
%\usepackage{epsfig} % for postscript graphics files
%\usepackage{mathptmx} % assumes new font selection scheme installed
%\usepackage{times} % assumes new font selection scheme installed
%\usepackage{amsmath} % assumes amsmath package installed
%\usepackage{amssymb}  % assumes amsmath package installed

\title{\LARGE \bf
	Which Node to Expand?
}

%\author{ \parbox{3 in}{\centering Huibert Kwakernaak*
%         \thanks{*Use the $\backslash$thanks command to put information here}\\
%         Faculty of Electrical Engineering, Mathematics and Computer Science\\
%         University of Twente\\
%         7500 AE Enschede, The Netherlands\\
%         {\tt\small h.kwakernaak@autsubmit.com}}
%         \hspace*{ 0.5 in}
%         \parbox{3 in}{ \centering Pradeep Misra**
%         \thanks{**The footnote marks may be inserted manually}\\
%        Department of Electrical Engineering \\
%         Wright State University\\
%         Dayton, OH 45435, USA\\
%         {\tt\small pmisra@cs.wright.edu}}
%}
\author{ {\bf Andrew Mitchell}  \\
	Department of Computer Science \\
	University of New Hampshire\\
	{\small ajx256@wildcats.unh.edu}
}
\date{\today}


\begin{document}



\maketitle
\thispagestyle{empty}
\pagestyle{empty}


%%%%%%%%%%%%%%%%%%%%%%%%%%%%%%%%%%%%%%%%%%%%%%%%%%%%%%%%%%%%%%%%%%%%%%%%%%%%%%%%
\begin{abstract}

The purpose of this research is to produce a real-time heuristic search algorithm prototype that reasons carefully about which node is best to expand next. Traditionally, nodes are selected for expansion so that the $f$-value of the selected node is minimized.

However, in real-time search the goal is not to find the minimal path from the start to the goal, it is to find the best path possible, towards the goal, within an allotted time or number of expansions per move. Therefore, in real-time search the node on the frontier with the lowest $f$-value is not necessarily the best node to expand. The best node to expand in this instance is the node that increases confidence the most about the best action that the agent can immediately take. 

I will implement a prototype algorithm that reasons more carefully about which node is best to expand next, and test it on some classic heuristic search benchmarks, as well as against traditional real-time search algorithms.
\end{abstract}

\end{document}